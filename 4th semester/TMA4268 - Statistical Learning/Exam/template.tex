% Options for packages loaded elsewhere
\PassOptionsToPackage{unicode}{hyperref}
\PassOptionsToPackage{hyphens}{url}
%
\documentclass[
]{article}
\title{Problem xy (use separate files for each problem)}
\author{10111}
\date{03 juni, 2022}

\usepackage{amsmath,amssymb}
\usepackage{lmodern}
\usepackage{iftex}
\ifPDFTeX
  \usepackage[T1]{fontenc}
  \usepackage[utf8]{inputenc}
  \usepackage{textcomp} % provide euro and other symbols
\else % if luatex or xetex
  \usepackage{unicode-math}
  \defaultfontfeatures{Scale=MatchLowercase}
  \defaultfontfeatures[\rmfamily]{Ligatures=TeX,Scale=1}
\fi
% Use upquote if available, for straight quotes in verbatim environments
\IfFileExists{upquote.sty}{\usepackage{upquote}}{}
\IfFileExists{microtype.sty}{% use microtype if available
  \usepackage[]{microtype}
  \UseMicrotypeSet[protrusion]{basicmath} % disable protrusion for tt fonts
}{}
\makeatletter
\@ifundefined{KOMAClassName}{% if non-KOMA class
  \IfFileExists{parskip.sty}{%
    \usepackage{parskip}
  }{% else
    \setlength{\parindent}{0pt}
    \setlength{\parskip}{6pt plus 2pt minus 1pt}}
}{% if KOMA class
  \KOMAoptions{parskip=half}}
\makeatother
\usepackage{xcolor}
\IfFileExists{xurl.sty}{\usepackage{xurl}}{} % add URL line breaks if available
\IfFileExists{bookmark.sty}{\usepackage{bookmark}}{\usepackage{hyperref}}
\hypersetup{
  pdftitle={Problem xy (use separate files for each problem)},
  pdfauthor={10111},
  hidelinks,
  pdfcreator={LaTeX via pandoc}}
\urlstyle{same} % disable monospaced font for URLs
\usepackage[margin=1in]{geometry}
\usepackage{color}
\usepackage{fancyvrb}
\newcommand{\VerbBar}{|}
\newcommand{\VERB}{\Verb[commandchars=\\\{\}]}
\DefineVerbatimEnvironment{Highlighting}{Verbatim}{commandchars=\\\{\}}
% Add ',fontsize=\small' for more characters per line
\usepackage{framed}
\definecolor{shadecolor}{RGB}{248,248,248}
\newenvironment{Shaded}{\begin{snugshade}}{\end{snugshade}}
\newcommand{\AlertTok}[1]{\textcolor[rgb]{0.94,0.16,0.16}{#1}}
\newcommand{\AnnotationTok}[1]{\textcolor[rgb]{0.56,0.35,0.01}{\textbf{\textit{#1}}}}
\newcommand{\AttributeTok}[1]{\textcolor[rgb]{0.77,0.63,0.00}{#1}}
\newcommand{\BaseNTok}[1]{\textcolor[rgb]{0.00,0.00,0.81}{#1}}
\newcommand{\BuiltInTok}[1]{#1}
\newcommand{\CharTok}[1]{\textcolor[rgb]{0.31,0.60,0.02}{#1}}
\newcommand{\CommentTok}[1]{\textcolor[rgb]{0.56,0.35,0.01}{\textit{#1}}}
\newcommand{\CommentVarTok}[1]{\textcolor[rgb]{0.56,0.35,0.01}{\textbf{\textit{#1}}}}
\newcommand{\ConstantTok}[1]{\textcolor[rgb]{0.00,0.00,0.00}{#1}}
\newcommand{\ControlFlowTok}[1]{\textcolor[rgb]{0.13,0.29,0.53}{\textbf{#1}}}
\newcommand{\DataTypeTok}[1]{\textcolor[rgb]{0.13,0.29,0.53}{#1}}
\newcommand{\DecValTok}[1]{\textcolor[rgb]{0.00,0.00,0.81}{#1}}
\newcommand{\DocumentationTok}[1]{\textcolor[rgb]{0.56,0.35,0.01}{\textbf{\textit{#1}}}}
\newcommand{\ErrorTok}[1]{\textcolor[rgb]{0.64,0.00,0.00}{\textbf{#1}}}
\newcommand{\ExtensionTok}[1]{#1}
\newcommand{\FloatTok}[1]{\textcolor[rgb]{0.00,0.00,0.81}{#1}}
\newcommand{\FunctionTok}[1]{\textcolor[rgb]{0.00,0.00,0.00}{#1}}
\newcommand{\ImportTok}[1]{#1}
\newcommand{\InformationTok}[1]{\textcolor[rgb]{0.56,0.35,0.01}{\textbf{\textit{#1}}}}
\newcommand{\KeywordTok}[1]{\textcolor[rgb]{0.13,0.29,0.53}{\textbf{#1}}}
\newcommand{\NormalTok}[1]{#1}
\newcommand{\OperatorTok}[1]{\textcolor[rgb]{0.81,0.36,0.00}{\textbf{#1}}}
\newcommand{\OtherTok}[1]{\textcolor[rgb]{0.56,0.35,0.01}{#1}}
\newcommand{\PreprocessorTok}[1]{\textcolor[rgb]{0.56,0.35,0.01}{\textit{#1}}}
\newcommand{\RegionMarkerTok}[1]{#1}
\newcommand{\SpecialCharTok}[1]{\textcolor[rgb]{0.00,0.00,0.00}{#1}}
\newcommand{\SpecialStringTok}[1]{\textcolor[rgb]{0.31,0.60,0.02}{#1}}
\newcommand{\StringTok}[1]{\textcolor[rgb]{0.31,0.60,0.02}{#1}}
\newcommand{\VariableTok}[1]{\textcolor[rgb]{0.00,0.00,0.00}{#1}}
\newcommand{\VerbatimStringTok}[1]{\textcolor[rgb]{0.31,0.60,0.02}{#1}}
\newcommand{\WarningTok}[1]{\textcolor[rgb]{0.56,0.35,0.01}{\textbf{\textit{#1}}}}
\usepackage{graphicx}
\makeatletter
\def\maxwidth{\ifdim\Gin@nat@width>\linewidth\linewidth\else\Gin@nat@width\fi}
\def\maxheight{\ifdim\Gin@nat@height>\textheight\textheight\else\Gin@nat@height\fi}
\makeatother
% Scale images if necessary, so that they will not overflow the page
% margins by default, and it is still possible to overwrite the defaults
% using explicit options in \includegraphics[width, height, ...]{}
\setkeys{Gin}{width=\maxwidth,height=\maxheight,keepaspectratio}
% Set default figure placement to htbp
\makeatletter
\def\fps@figure{htbp}
\makeatother
\setlength{\emergencystretch}{3em} % prevent overfull lines
\providecommand{\tightlist}{%
  \setlength{\itemsep}{0pt}\setlength{\parskip}{0pt}}
\setcounter{secnumdepth}{-\maxdimen} % remove section numbering
\ifLuaTeX
  \usepackage{selnolig}  % disable illegal ligatures
\fi

\begin{document}
\maketitle

The following packages are probably needed to solve the exam, but there
might be other packages that you prefer to use. \textbf{Please make sure
that you have installed at least the following R packages BEFORE the
exam starts}! If you want to be sure that you don't miss any R package,
you can also go through all the exercises we did in the course and
install all the packages we used.

\begin{Shaded}
\begin{Highlighting}[]
\FunctionTok{install.packages}\NormalTok{(}\StringTok{"knitr"}\NormalTok{)}
\FunctionTok{install.packages}\NormalTok{(}\StringTok{"MASS"}\NormalTok{)}
\FunctionTok{install.packages}\NormalTok{(}\StringTok{"caret"}\NormalTok{)}
\FunctionTok{install.packages}\NormalTok{(}\StringTok{"pls"}\NormalTok{)}
\FunctionTok{install.packages}\NormalTok{(}\StringTok{"glmnet"}\NormalTok{)}
\FunctionTok{install.packages}\NormalTok{(}\StringTok{"gam"}\NormalTok{)}
\FunctionTok{install.packages}\NormalTok{(}\StringTok{"gbm"}\NormalTok{)}
\FunctionTok{install.packages}\NormalTok{(}\StringTok{"randomForest"}\NormalTok{)}
\FunctionTok{install.packages}\NormalTok{(}\StringTok{"ggfortify"}\NormalTok{)}
\FunctionTok{intall.packages}\NormalTok{(}\StringTok{"leaps"}\NormalTok{)}
\FunctionTok{install.packages}\NormalTok{(}\StringTok{"pROC"}\NormalTok{)}
\end{Highlighting}
\end{Shaded}

You can start with an R-chunck where you load some R packages, for
example as follows (replace with the R packages you need to solve the
task). But replace the chunc option by \texttt{echo=FALSE}, because it
is not something you need to print in the pdf you upload.

\begin{Shaded}
\begin{Highlighting}[]
\FunctionTok{library}\NormalTok{(knitr)}
\FunctionTok{library}\NormalTok{(MASS)}
\FunctionTok{library}\NormalTok{(keras)}
\FunctionTok{library}\NormalTok{(caret)}
\FunctionTok{library}\NormalTok{(pls)}
\FunctionTok{library}\NormalTok{(glmnet)}
\FunctionTok{library}\NormalTok{(gam)}
\FunctionTok{library}\NormalTok{(gbm)}
\FunctionTok{library}\NormalTok{(randomForest)}
\FunctionTok{library}\NormalTok{(ggfortify)}
\FunctionTok{library}\NormalTok{(leaps)}
\FunctionTok{library}\NormalTok{(pROC)}
\end{Highlighting}
\end{Shaded}

Here are some examples how to include R code (required) and LaTex
formulas (recommended), in the same way we did it during the whole
semester. Please install latex on your computer if you would like to
include mathematical formulas. The alternative is that you write
mathematical things on paper, your iPad or similar, and upload
separately.

To prepare for the exam you might also want to go through the bonus part
about R Markdown in the online R course:

\url{https://digit.ntnu.no/courses/course-v1:NTNU+IMF001+2020/course/}

Here is a code chunk taken from Compulsory 1 of 2021 (replace it with
the code you need in the exam):

\begin{Shaded}
\begin{Highlighting}[]
\NormalTok{id }\OtherTok{\textless{}{-}} \StringTok{"1nLen1ckdnX4P9n8ShZeU7zbXpLc7qiwt"} \CommentTok{\# google file ID}
\NormalTok{d.worm }\OtherTok{\textless{}{-}} \FunctionTok{read.csv}\NormalTok{(}\FunctionTok{sprintf}\NormalTok{(}\StringTok{"https://docs.google.com/uc?id=\%s\&export=download"}\NormalTok{, id))}
\FunctionTok{head}\NormalTok{(d.worm)}
\end{Highlighting}
\end{Shaded}

\begin{verbatim}
##   Gattung Nummer GEWICHT FANGDATUM MAGENUMF
## 1      Oc     32    0.19  23.09.97     1.56
## 2      Oc     34    0.59  23.09.97     1.63
## 3      Oc     48    0.09  23.09.97     1.69
## 4      Oc     55    0.23  23.09.97     1.69
## 5      Oc     41    0.24  23.09.97     1.75
## 6      Oc     24    0.19  23.09.97     1.81
\end{verbatim}

\hypertarget{a}{%
\subsection{a)}\label{a}}

R code, results and answers to sub-question a)

\hypertarget{b}{%
\subsection{b)}\label{b}}

Below you have to complete the code and then replace \texttt{eval=FALSE}
by \texttt{eval=TRUE} in the chunk options:

\begin{Shaded}
\begin{Highlighting}[]
\FunctionTok{ggplot}\NormalTok{(d.worm,}\FunctionTok{aes}\NormalTok{(}\AttributeTok{x=}\NormalTok{ ... ,}\AttributeTok{y=}\NormalTok{  ... ,}\AttributeTok{colour=}\NormalTok{ ...)) }\SpecialCharTok{+} \FunctionTok{geom\_point}\NormalTok{() }\SpecialCharTok{+} \FunctionTok{theme\_bw}\NormalTok{()}
\end{Highlighting}
\end{Shaded}

Note that the default figure width and height have been set globally as
\texttt{fig.width=4,\ fig.height=3}, but if you would like to change
that (e.g., due to space constraints), you can include a different width
and height directly into the chunk options, again using
\texttt{fig.width=...,\ fig.height=...}.

\hypertarget{c}{%
\subsection{c)}\label{c}}

Here is an example with LaTeX code:
\(y_i = \beta_0 + \beta_1 x_i + \epsilon_i\), where \(\beta_0=1\) and
\(\beta_1= 2\). In display mode you use for example

\[y_i = \beta_0 + \beta_1 x_i + \epsilon_i \ , \epsilon_i \sim \mathcal{N}(0,\sigma^2) \ .\]

\end{document}
