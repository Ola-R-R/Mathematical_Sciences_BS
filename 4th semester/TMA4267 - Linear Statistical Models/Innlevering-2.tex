% Options for packages loaded elsewhere
\PassOptionsToPackage{unicode}{hyperref}
\PassOptionsToPackage{hyphens}{url}
%
\documentclass[
]{article}
\title{Assignment 2}
\author{Ola Rasmussen}
\date{}

\usepackage{amsmath,amssymb}
\usepackage{lmodern}
\usepackage{iftex}
\ifPDFTeX
  \usepackage[T1]{fontenc}
  \usepackage[utf8]{inputenc}
  \usepackage{textcomp} % provide euro and other symbols
\else % if luatex or xetex
  \usepackage{unicode-math}
  \defaultfontfeatures{Scale=MatchLowercase}
  \defaultfontfeatures[\rmfamily]{Ligatures=TeX,Scale=1}
\fi
% Use upquote if available, for straight quotes in verbatim environments
\IfFileExists{upquote.sty}{\usepackage{upquote}}{}
\IfFileExists{microtype.sty}{% use microtype if available
  \usepackage[]{microtype}
  \UseMicrotypeSet[protrusion]{basicmath} % disable protrusion for tt fonts
}{}
\makeatletter
\@ifundefined{KOMAClassName}{% if non-KOMA class
  \IfFileExists{parskip.sty}{%
    \usepackage{parskip}
  }{% else
    \setlength{\parindent}{0pt}
    \setlength{\parskip}{6pt plus 2pt minus 1pt}}
}{% if KOMA class
  \KOMAoptions{parskip=half}}
\makeatother
\usepackage{xcolor}
\IfFileExists{xurl.sty}{\usepackage{xurl}}{} % add URL line breaks if available
\IfFileExists{bookmark.sty}{\usepackage{bookmark}}{\usepackage{hyperref}}
\hypersetup{
  pdftitle={Assignment 2},
  pdfauthor={Ola Rasmussen},
  hidelinks,
  pdfcreator={LaTeX via pandoc}}
\urlstyle{same} % disable monospaced font for URLs
\usepackage[margin=1in]{geometry}
\usepackage{color}
\usepackage{fancyvrb}
\newcommand{\VerbBar}{|}
\newcommand{\VERB}{\Verb[commandchars=\\\{\}]}
\DefineVerbatimEnvironment{Highlighting}{Verbatim}{commandchars=\\\{\}}
% Add ',fontsize=\small' for more characters per line
\usepackage{framed}
\definecolor{shadecolor}{RGB}{248,248,248}
\newenvironment{Shaded}{\begin{snugshade}}{\end{snugshade}}
\newcommand{\AlertTok}[1]{\textcolor[rgb]{0.94,0.16,0.16}{#1}}
\newcommand{\AnnotationTok}[1]{\textcolor[rgb]{0.56,0.35,0.01}{\textbf{\textit{#1}}}}
\newcommand{\AttributeTok}[1]{\textcolor[rgb]{0.77,0.63,0.00}{#1}}
\newcommand{\BaseNTok}[1]{\textcolor[rgb]{0.00,0.00,0.81}{#1}}
\newcommand{\BuiltInTok}[1]{#1}
\newcommand{\CharTok}[1]{\textcolor[rgb]{0.31,0.60,0.02}{#1}}
\newcommand{\CommentTok}[1]{\textcolor[rgb]{0.56,0.35,0.01}{\textit{#1}}}
\newcommand{\CommentVarTok}[1]{\textcolor[rgb]{0.56,0.35,0.01}{\textbf{\textit{#1}}}}
\newcommand{\ConstantTok}[1]{\textcolor[rgb]{0.00,0.00,0.00}{#1}}
\newcommand{\ControlFlowTok}[1]{\textcolor[rgb]{0.13,0.29,0.53}{\textbf{#1}}}
\newcommand{\DataTypeTok}[1]{\textcolor[rgb]{0.13,0.29,0.53}{#1}}
\newcommand{\DecValTok}[1]{\textcolor[rgb]{0.00,0.00,0.81}{#1}}
\newcommand{\DocumentationTok}[1]{\textcolor[rgb]{0.56,0.35,0.01}{\textbf{\textit{#1}}}}
\newcommand{\ErrorTok}[1]{\textcolor[rgb]{0.64,0.00,0.00}{\textbf{#1}}}
\newcommand{\ExtensionTok}[1]{#1}
\newcommand{\FloatTok}[1]{\textcolor[rgb]{0.00,0.00,0.81}{#1}}
\newcommand{\FunctionTok}[1]{\textcolor[rgb]{0.00,0.00,0.00}{#1}}
\newcommand{\ImportTok}[1]{#1}
\newcommand{\InformationTok}[1]{\textcolor[rgb]{0.56,0.35,0.01}{\textbf{\textit{#1}}}}
\newcommand{\KeywordTok}[1]{\textcolor[rgb]{0.13,0.29,0.53}{\textbf{#1}}}
\newcommand{\NormalTok}[1]{#1}
\newcommand{\OperatorTok}[1]{\textcolor[rgb]{0.81,0.36,0.00}{\textbf{#1}}}
\newcommand{\OtherTok}[1]{\textcolor[rgb]{0.56,0.35,0.01}{#1}}
\newcommand{\PreprocessorTok}[1]{\textcolor[rgb]{0.56,0.35,0.01}{\textit{#1}}}
\newcommand{\RegionMarkerTok}[1]{#1}
\newcommand{\SpecialCharTok}[1]{\textcolor[rgb]{0.00,0.00,0.00}{#1}}
\newcommand{\SpecialStringTok}[1]{\textcolor[rgb]{0.31,0.60,0.02}{#1}}
\newcommand{\StringTok}[1]{\textcolor[rgb]{0.31,0.60,0.02}{#1}}
\newcommand{\VariableTok}[1]{\textcolor[rgb]{0.00,0.00,0.00}{#1}}
\newcommand{\VerbatimStringTok}[1]{\textcolor[rgb]{0.31,0.60,0.02}{#1}}
\newcommand{\WarningTok}[1]{\textcolor[rgb]{0.56,0.35,0.01}{\textbf{\textit{#1}}}}
\usepackage{graphicx}
\makeatletter
\def\maxwidth{\ifdim\Gin@nat@width>\linewidth\linewidth\else\Gin@nat@width\fi}
\def\maxheight{\ifdim\Gin@nat@height>\textheight\textheight\else\Gin@nat@height\fi}
\makeatother
% Scale images if necessary, so that they will not overflow the page
% margins by default, and it is still possible to overwrite the defaults
% using explicit options in \includegraphics[width, height, ...]{}
\setkeys{Gin}{width=\maxwidth,height=\maxheight,keepaspectratio}
% Set default figure placement to htbp
\makeatletter
\def\fps@figure{htbp}
\makeatother
\setlength{\emergencystretch}{3em} % prevent overfull lines
\providecommand{\tightlist}{%
  \setlength{\itemsep}{0pt}\setlength{\parskip}{0pt}}
\setcounter{secnumdepth}{-\maxdimen} % remove section numbering
\ifLuaTeX
  \usepackage{selnolig}  % disable illegal ligatures
\fi

\begin{document}
\maketitle

\hypertarget{problem-1}{%
\subsection{Problem 1}\label{problem-1}}

\begin{enumerate}
\def\labelenumi{\alph{enumi}.}
\item
  \begin{enumerate}
  \def\labelenumii{\arabic{enumii}.}
  \item
    \begin{itemize}
    \tightlist
    \item
      Estimate: \(\hat\beta = (X^TX)^{-1}X^TY\)
    \item
      Std. Error: \(\dfrac{\sigma}{\sqrt{n}}\)
    \item
      T-value: \(T = \dfrac{Estimate}{Std. Error}\)
    \item
      Pr(\textgreater\textbar t\textbar): \(P-value(t)=P(T\geq t)\)
    \item
      \(Y = X\beta+\epsilon\) is a quantitative measurement of disease
      progression one year after baseline.
    \item
      \(n\) is the number of observations and \(\sigma\) is standard
      deviation.
    \end{itemize}
  \item
    Estimate of the intercept we interpret is at \(\hat\beta_0\). We get
    this when all the other covariates are zero.
  \item
    When the bmi covariate increases by 1, the Y-value increases by
    5.59548.
  \item
    Residual standard error: 54.16 on 431 degrees of freedom.

    The formula is: \(\dfrac{1}{n}\sum_{i=1}^\infty(Y_i - \hat Y_i)^2\)
  \item
    With a significant level at 0.05 we would consider sex, bmi, map and
    ltg so be significant.

    \(H_0: p-value_{covariate}\leq\alpha\)

    versus

    \(H_1: p-value_{covariate}>\alpha\)

    For the p-value to be valid, we need to assume that the
    null-hypothesis is correct.
  \end{enumerate}
\item
  I would say that the fit of the full model is OK. The adjusted
  R-squared value is about 0.5, and since humans are hard to predict,
  that is a good adjusted R-squared value.

  Less than half of the null-hypotheses are rejected, so i would not say
  that the model is significant at level \(\alpha = 0.05\).

  \(H_0: \beta_{age} = \beta_{sex} =\beta_{bmi} =\beta_{map} =\beta_{tc} =\beta_{ldl} =\beta_{hdl} =\beta_{tch} =\beta_{ltg} =\beta_{glu} = 0\)

  versus

  \(H_1:\) at least one \(\ne 0\)

  Multiple R-squared is a measurement for the fit of the model. The
  value 0.5176 means that the full model explains roughly 50\% of the
  data.
\item
  A reduced model can have a better performance than a full model
  because it might reduce overfitting when we want to predict the
  future.

  In the best subset model selection, all the possible combinations of
  the independent variables are considered. They are tested by some
  criterion.

  Some of these criterion are adjusted R-squared and Bayesian
  Information Criterion (BIC).

  Adjusted R-squared is used because it includes a correction term for
  the number of parameters. The bigger the better.

  BIC introduces a penalty term for the number of parameters. The lower
  the better.

  Based on the results of the adjusted R-squared and the BIC criteria, I
  choose model 6. It has the second lowest BIC value while also having
  the second highest adjusted R-squared value. The other contenders are
  model 5 and 7. Model 5 has a much lower adjusted R-squared value and
  model 7 has a much higher BIC value.

\begin{Shaded}
\begin{Highlighting}[]
\NormalTok{ds }\OtherTok{\textless{}{-}} \FunctionTok{read.csv}\NormalTok{(}\StringTok{"https://web.stanford.edu/\textasciitilde{}hastie/CASI\_files/DATA/diabetes.csv"}\NormalTok{, }\AttributeTok{sep =} \StringTok{","}\NormalTok{)}
\NormalTok{model6 }\OtherTok{\textless{}{-}} \FunctionTok{lm}\NormalTok{(prog }\SpecialCharTok{\textasciitilde{}}\NormalTok{ sex }\SpecialCharTok{+}\NormalTok{ bmi }\SpecialCharTok{+}\NormalTok{ map }\SpecialCharTok{+}\NormalTok{ tc }\SpecialCharTok{+}\NormalTok{ ldl }\SpecialCharTok{+}\NormalTok{ ltg, }\AttributeTok{data =}\NormalTok{ ds)}
\FunctionTok{summary}\NormalTok{(model6)}
\end{Highlighting}
\end{Shaded}

\begin{verbatim}
## 
## Call:
## lm(formula = prog ~ sex + bmi + map + tc + ldl + ltg, data = ds)
## 
## Residuals:
##      Min       1Q   Median       3Q      Max 
## -158.277  -39.476   -2.068   37.221  148.693 
## 
## Coefficients:
##              Estimate Std. Error t value Pr(>|t|)    
## (Intercept) -335.3586    25.3234 -13.243  < 2e-16 ***
## sex          -21.5914     5.7056  -3.784 0.000176 ***
## bmi            5.7110     0.7073   8.075 6.69e-15 ***
## map            1.1266     0.2158   5.219 2.79e-07 ***
## tc            -1.0429     0.2208  -4.724 3.12e-06 ***
## ldl            0.8433     0.2298   3.670 0.000272 ***
## ltg          168.7953    16.8279  10.031  < 2e-16 ***
## ---
## Signif. codes:  0 '***' 0.001 '**' 0.01 '*' 0.05 '.' 0.1 ' ' 1
## 
## Residual standard error: 54.06 on 435 degrees of freedom
## Multiple R-squared:  0.5149,  Adjusted R-squared:  0.5082 
## F-statistic: 76.95 on 6 and 435 DF,  p-value: < 2.2e-16
\end{verbatim}

  Comparing Model 6 with the full model we observe that the adjusted
  R-squared value has increased, implying that the model fit more to the
  data. We also observe that all of the null-hypotheses are rejected,
  which means that the model is significant.
\item
  Test:

  \(H_0: \beta_{age} = \beta_{tc} =\beta_{ldl} =\beta_{tch} =\beta_{glu} = 0\)

  versus

  \(H_1:\) at least one \(\ne 0\)

\begin{Shaded}
\begin{Highlighting}[]
\NormalTok{ds }\OtherTok{\textless{}{-}} \FunctionTok{read.csv}\NormalTok{(}\StringTok{"https://web.stanford.edu/\textasciitilde{}hastie/CASI\_files/DATA/diabetes.csv"}\NormalTok{, }\AttributeTok{sep =} \StringTok{","}\NormalTok{)}
\NormalTok{reduced }\OtherTok{\textless{}{-}} \FunctionTok{lm}\NormalTok{(prog }\SpecialCharTok{\textasciitilde{}}\NormalTok{ age }\SpecialCharTok{+}\NormalTok{ tc }\SpecialCharTok{+}\NormalTok{ ldl }\SpecialCharTok{+}\NormalTok{ tch }\SpecialCharTok{+}\NormalTok{ glu, }\AttributeTok{data =}\NormalTok{ ds)}
\FunctionTok{summary}\NormalTok{(reduced)}
\end{Highlighting}
\end{Shaded}

\begin{verbatim}
## 
## Call:
## lm(formula = prog ~ age + tc + ldl + tch + glu, data = ds)
## 
## Residuals:
##     Min      1Q  Median      3Q     Max 
## -156.22  -51.39   -7.70   47.15  191.84 
## 
## Coefficients:
##             Estimate Std. Error t value Pr(>|t|)    
## (Intercept) -94.1711    28.0725  -3.355 0.000864 ***
## age           0.3166     0.2547   1.243 0.214455    
## tc            0.7294     0.2128   3.428 0.000666 ***
## ldl          -1.3202     0.2669  -4.946 1.08e-06 ***
## tch          29.9553     3.4764   8.617  < 2e-16 ***
## glu           1.3528     0.3140   4.308 2.03e-05 ***
## ---
## Signif. codes:  0 '***' 0.001 '**' 0.01 '*' 0.05 '.' 0.1 ' ' 1
## 
## Residual standard error: 65.7 on 436 degrees of freedom
## Multiple R-squared:  0.2819,  Adjusted R-squared:  0.2737 
## F-statistic: 34.23 on 5 and 436 DF,  p-value: < 2.2e-16
\end{verbatim}

  We can see that the adjusted R-squared value has drastically gone
  down, so i would prefer the full model of this reduces model.
\end{enumerate}

\hypertarget{problem-2}{%
\subsection{Problem 2}\label{problem-2}}

\begin{enumerate}
\def\labelenumi{\alph{enumi}.}
\item
\begin{Shaded}
\begin{Highlighting}[]
\NormalTok{pvalues }\OtherTok{\textless{}{-}} \FunctionTok{scan}\NormalTok{(}\StringTok{"https://www.math.ntnu.no/emner/TMA4267/2018v/pvalues.txt"}\NormalTok{)}
\FunctionTok{sum}\NormalTok{(pvalues }\SpecialCharTok{\textless{}} \FloatTok{0.05}\NormalTok{)}
\end{Highlighting}
\end{Shaded}

\begin{verbatim}
## [1] 155
\end{verbatim}

  We reject 155 null-hypotheses.

  A type 1 error occurs when we reject a null-hypothesis when said
  null-hypothesis is true.

  We do not know the number of false positive findings in out data, but
  we can assume it given \(\alpha = 0.05\)
\item
  The familywise error rate (FWER) is defined as the probability of one
  or more false positive findings.

  Controlling the FWER at level 0.05 means that we set an upper limit to
  the FWER. That is we set a cut-off on the p-value.

  Given \(\alpha = 0.05\) and \(m = 1000\), we want the cut-off on
  p-values to be
  \(\alpha_{loc} = \dfrac{\alpha}{m} = \dfrac{0.05}{1000} = 0.00005\)
  for out data using the Bonferroni method.

\begin{Shaded}
\begin{Highlighting}[]
\NormalTok{pvalues }\OtherTok{\textless{}{-}} \FunctionTok{scan}\NormalTok{(}\StringTok{"https://www.math.ntnu.no/emner/TMA4267/2018v/pvalues.txt"}\NormalTok{)}
\FunctionTok{sum}\NormalTok{(pvalues }\SpecialCharTok{\textless{}} \FloatTok{0.00005}\NormalTok{)}
\end{Highlighting}
\end{Shaded}

\begin{verbatim}
## [1] 50
\end{verbatim}

  Using the Bonferroni method with \(\alpha_{loc} = 0.00005\), we reject
  50 null-hypotheses.
\item
  Assuming that the first 900 null-hypotheses are true ant the last 100
  are false, it implies that 10\% of the number of type 1 and type 2
  errors are false.
\end{enumerate}

\end{document}
